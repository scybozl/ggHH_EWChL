\documentclass[a4paper]{article}
\usepackage{axodraw}
\usepackage{longtable}
\usepackage{listings}
\usepackage{amsmath}
\usepackage{makeidx}
\usepackage{hyperref}

\newcommand{\bra}[1]{\langle #1 \vert}
\newcommand{\brb}[1]{[ #1 \vert}
\newcommand{\kea}[1]{\vert #1 \rangle}
\newcommand{\keb}[1]{\vert #1 ]}
\newcommand{\Spaa}[1]{\langle #1 \rangle}
\newcommand{\Spab}[1]{\langle #1]}
\newcommand{\Spba}[1]{[ #1 \rangle}
\newcommand{\Spbb}[1]{[ #1 ]}

\allowdisplaybreaks[1]

\title{\texttt{GoSam 2.0.4}: ${g}{g}\rightarrow{h}{h}$}
\author{lscyboz}
\date{2019-06-17 (12:55:12)}

\renewcommand{\indexname}{Index of all Loop Diagrams}

\makeindex
\begin{document}
\maketitle
\begin{abstract}
\noindent This process consists of 2 tree-level diagrams and no NLO diagrams. GoSam has identified no different groups of NLO diagrams by analyzing their one-loop integrals.
\end{abstract}
\newpage
\tableofcontents
\newpage

\section{Helicities}

\begin{longtable}[c]{r|cccc}
\bf{Index} &1&2&3&4\\
\hline
\endfirsthead
\bf{Index} &1&2&3&4\\
\hline
\endhead 
$0$& $-$& $-$& $0$& $0$\\
$1\rightarrow 0$& $+$& $-$& $0$& $0$\\
$2\rightarrow 0$& $-$& $+$& $0$& $0$\\
$3\rightarrow 0$& $+$& $+$& $0$& $0$\\
\end{longtable}
\section{Wave Functions}
In this section, we use $l_i=k_i$ for massless particles;
in spinors $\kea{i}$ (resp. $\keb{i}$) denote $\kea{l_i}$ (resp. $\keb{l_i}$).
For the massive particles we have:
\begin{align}
l_{3} &= k_{3} - \frac{mdlMh^2}{%
      2 k_{3}\cdot k_{2}}k_{2}\\
l_{4} &= k_{4} - \frac{mdlMh^2}{%
      2 k_{4}\cdot k_{2}}k_{2}
\end{align}

All helicity amplitudes are defined in terms of the following wave functions:
\begin{itemize}
\item $g(k_{1})$ 
% incoming vector particle
\begin{align}
\varepsilon^\mu_+(k_{1}) &=
   \frac{\Spab{2\vert\gamma^\mu\vert 1}}{%
   \sqrt{2}\Spaa{2\vert 1}}\\
\varepsilon^\mu_-(k_{1}) &=
   \frac{\Spba{2\vert\gamma^\mu\vert 1}}{%
   \sqrt{2}\Spbb{1\vert 2}}
\end{align}
\item $g(k_{2})$ 
% incoming vector particle
\begin{align}
\varepsilon^\mu_+(k_{2}) &=
   \frac{\Spab{1\vert\gamma^\mu\vert 2}}{%
   \sqrt{2}\Spaa{1\vert 2}}\\
\varepsilon^\mu_-(k_{2}) &=
   \frac{\Spba{1\vert\gamma^\mu\vert 2}}{%
   \sqrt{2}\Spbb{2\vert 1}}
\end{align}
\item $h(k_3)$ 
\begin{align}
% outgoing scalar particle
\epsilon(k_{3}) &= 1
\end{align}
\item $h(k_4)$ 
\begin{align}
% outgoing scalar particle
\epsilon(k_{4}) &= 1
\end{align}
\end{itemize}

%------------------------------------------------------------------------
\section{Colour Basis}
\begin{align}
\vert c_{1}\rangle &=g^{A_{1}}_{(1)}g^{A_{2}}_{(2)}\textrm{tr}\left\{T^{A_{2}}T^{A_{1}}\right\}
\end{align}


\section{Tree Diagrams}
\lstinputlisting[title={QGraf Setup},frame=tlrb]{../diagrams-0.log}

\begin{longtable}{cc}
\endfirsthead
\endhead
%---#[ tree diagram1:
\hbox{
\begin{minipage}{0.45\textwidth}
\begin{center}
% Diagram 1:
\begin{picture}(140,120)(-10,-10)
   \Gluon(102.4,85.4)(62.1,54.2){3}{10} % part21-propagator
   \Text(104.3,87.7)[lb]{$g(k_{1})$}
   \Gluon(109.9,22.2)(62.1,54.2){3}{12} % part21-propagator
   \Text(108.2,24.7)[lt]{$g(k_{2})$}
   \DashLine(62.1,54.2)(37.0,3.8){5} % part25-propagator
   \Text(39.7,5.1)[rt]{$h(k_{3})$}
   \DashLine(62.1,54.2)(1.2,59.8){5} % part25-propagator
   \Text(0.9,62.7)[rb]{$h(k_{4})$}
   \Vertex(62.1,54.2){3} % part21-part21-part25-part25 vertex
\end{picture}
\\
{\sl Diagram~1}
\end{center}
\end{minipage}}
%---#] tree diagram1:
&
%---#[ tree diagram2:
\hbox{
\begin{minipage}{0.45\textwidth}
\begin{center}
% Diagram 2:
\begin{picture}(140,120)(-10,-10)
   \Gluon(102.4,85.4)(58.0,69.2){3}{9} % part21-propagator
   \Text(103.5,88.2)[lb]{$g(k_{1})$}
   \Gluon(1.2,59.8)(58.0,69.2){3}{12} % part21-propagator
   \Text(0.7,56.8)[rt]{$g(k_{2})$}
   \DashLine(68.3,31.7)(109.9,22.2){5} % part25-propagator
   \Text(110.6,19.3)[lt]{$h(k_{3})$}
   \DashLine(68.3,31.7)(37.0,3.8){5} % part25-propagator
   \Text(39.0,6.0)[rt]{$h(k_{4})$}
   \Vertex(58.0,69.2){3} % part21-part21-part25 vertex
   \Vertex(68.3,31.7){3} % part25-part25-part25 vertex
   \DashLine(68.3,31.7)(58.0,69.2){5} % part25-propagator
   \Text(60.2,49.7)[rt]{$h$}
\end{picture}
\\
{\sl Diagram~2}
\end{center}
\end{minipage}}
%---#] tree diagram2:
\end{longtable}




\printindex

\section{Related Work}
If you publish results obtained by using this matrix element code
please cite the appropriate papers in the bibliography of this document.

Scientific publications prepared using the present version of
\textsc{GoSam} or any modified version of it or any code linking to
\textsc{GoSam} or parts of it should make a clear
reference to the publications~\cite{Cullen:2014yla,Cullen:2011ac}.

For graph generation we use QGraf~\cite{Nogueira:1991ex}.
The Feynman diagrams are further processed with the symbolic manipulation
program FORM~\cite{Kuipers:2012rf,Vermaseren:2000nd} using the FORM library
SPINNEY~\cite{Cullen:2010jv}.
The Fortran~90 code is generated using
FORM~\cite{Kuipers:2012rf,Vermaseren:2000nd}.



Please, make sure, you also give credit to the authors of the scalar
loop libraries, if you configured the amplitude code such that it calls
other libraries than the ones mentioned so far. Depending on your
configuration you might use one or more of the following programs for
the evaluation of the scalar integrals:
\begin{itemize}
\item OneLOop~\cite{vanHameren:2010cp},
\item QCDLoop~\cite{Ellis:2007qk}, which uses FF~\cite{vanOldenborgh:1990yc},
\item LoopTools~\cite{Hahn:1998yk}, which uses FF~\cite{vanOldenborgh:1990yc}.
\item GOLEM95~\cite{Binoth:2008uq,Guillet:2013msa} which uses OneLOop~\cite{vanHameren:2010cp}
   and may be configured such that it uses
   LoopTools~\cite{Hahn:1998yk,vanOldenborgh:1990yc}.
\end{itemize}

\begin{thebibliography}{ABC}
%\cite{Cullen:2014yla}
\bibitem{Cullen:2014yla}
  G.~Cullen, H.~van Deurzen, N.~Greiner, G.~Heinrich, G.~Luisoni, P.~Mastrolia, E.~Mirabella and G.~Ossola {\it et al.},
  ``GoSam-2.0: a tool for automated one-loop calculations within the Standard Model and beyond,''
  Eur.\ Phys.\ J.\ C {\bf 74} (2014) 8,  3001
  [\href{http://arxiv.org/abs/1404.7096}{arXiv:1404.7096 [hep-ph]}].
  %%CITATION = ARXIV:1404.7096;%%
%\cite{Cullen:2011ac}
\bibitem{Cullen:2011ac}
  G.~Cullen, N.~Greiner, G.~Heinrich, G.~Luisoni, P.~Mastrolia, G.~Ossola, T.~Reiter and F.~Tramontano,
  ``Automated One-Loop Calculations with GoSam,''
  Eur.\ Phys.\ J.\ C {\bf 72} (2012) 1889
  [\href{http://arxiv.org/abs/1111.2034}{arXiv:1111.2034 [hep-ph]}].
  %%CITATION = ARXIV:1111.2034;%%
%\cite{Nogueira:1991ex}
\bibitem{Nogueira:1991ex}
  P.~Nogueira,
  ``Automatic Feynman graph generation,''
  J.\ Comput.\ Phys.\  {\bf 105} (1993) 279.
  %%CITATION = JCTPA,105,279;%%
%\cite{Kuipers:2012rf}
\bibitem{Kuipers:2012rf}
  J.~Kuipers, T.~Ueda, J.~A.~M.~Vermaseren and J.~Vollinga,
  ``FORM version 4.0,''
  Comput.\ Phys.\ Commun.\  {\bf 184} (2013) 1453
  [\href{http://arxiv.org/abs/1203.6543}{arXiv:1203.6543 [cs.SC]}].
  %%CITATION = ARXIV:1203.6543;%%
%\cite{Vermaseren:2000nd}
\bibitem{Vermaseren:2000nd}
  J.~A.~M.~Vermaseren,
  ``New features of FORM,''
  arXiv:math-ph/0010025.
  %%CITATION = MATH-PH/0010025;%%
%\cite{Cullen:2010jv}
\bibitem{Cullen:2010jv}
  G.~Cullen, M.~Koch-Janusz and T.~Reiter,
  ``spinney: A Form Library for Helicity Spinors,''
  \href{http://arxiv.org/abs/1008.0803}{arXiv:1008.0803 [hep-ph]}.
  %%CITATION = ARXIV:1008.0803;%%
%\cite{Reiter:2009ts}
%\cite{Guillet:2013msa}
\bibitem{Guillet:2013msa}
  J.~P.~Guillet, G.~Heinrich and J.~F.~von Soden-Fraunhofen,
  ``Tools for NLO automation: extension of the golem95C integral library,''
  Comput.\ Phys.\ Commun.\  {\bf 185} (2014) 1828
  [\href{http://arxiv.org/abs/1312.3887}{arXiv:1312.3887 [hep-ph]}].
  %%CITATION = ARXIV:1312.3887;%%
%\cite{Binoth:2008uq}
\bibitem{Binoth:2008uq}
  T.~Binoth, J.~P.~Guillet, G.~Heinrich, E.~Pilon and T.~Reiter,
  ``Golem95: a numerical program to calculate one-loop tensor integrals with up
  to six external legs,''
  Comput.\ Phys.\ Commun.\  {\bf 180} (2009) 2317
  [\href{http://arxiv.org/abs/0810.0992}{arXiv:0810.0992 [hep-ph]}].
  %%CITATION = CPHCB,180,2317;%%
%\cite{Cullen:2011kv}
\bibitem{Cullen:2011kv}
  G.~Cullen, J.~P.~.Guillet, G.~Heinrich, T.~Kleinschmidt, E.~Pilon, T.~Reiter, M.~Rodgers,
  ``Golem95C: A library for one-loop integrals with complex masses,''
  Comput.\ Phys.\ Commun.\  {\bf 182 } (2011)  2276-2284.
  [\href{http://arxiv.org/abs/1101.5595}{arXiv:1101.5595 [hep-ph]}].
%\cite{vanHameren:2010cp}
\bibitem{vanHameren:2010cp}
  A.~van Hameren,
  ``OneLOop: For the evaluation of one-loop scalar functions,''
  [\href{http://arxiv.org/abs/1007.4716}{arXiv:1007.4716 [hep-ph]}].
%\cite{Ellis:2007qk}
\bibitem{Ellis:2007qk}
  R.~K.~Ellis, G.~Zanderighi,
  ``Scalar one-loop integrals for QCD,''
  JHEP {\bf 0802 } (2008)  002.
  [\href{http://arxiv.org/abs/0712.1851}{arXiv:0712.1851 [hep-ph]}].
%\cite{vanOldenborgh:1990yc}
\bibitem{vanOldenborgh:1990yc}
  G.~J.~van Oldenborgh,
  ``FF: A Package to evaluate one loop Feynman diagrams,''
  Comput.\ Phys.\ Commun.\  {\bf 66 } (1991)  1-15.
%\cite{Hahn:1998yk}
\bibitem{Hahn:1998yk}
  T.~Hahn, M.~Perez-Victoria,
  ``Automatized one loop calculations in four-dimensions and D-dimensions,''
  Comput.\ Phys.\ Commun.\  {\bf 118 } (1999)  153-165.
  [hep-ph/9807565].
%\cite{Heinrich:2010ax}
\bibitem{Heinrich:2010ax}
  G.~Heinrich, G.~Ossola, T.~Reiter, F.~Tramontano,
  ``Tensorial Reconstruction at the Integrand Level,''
  JHEP {\bf 1010 } (2010)  105.
  [\href{http://arxiv.org/abs/1008.2441}{arXiv:1008.2441 [hep-ph]}].
\end{thebibliography}
\end{document}
