
The results were obtained using the
PDF4LHC15{\tt\_}nlo{\tt\_}30{\tt\_}pdfas~\cite{Butterworth:2015oua,CT14,MMHT14,NNPDF}
parton distribution functions interfaced to our code via
LHAPDF~\cite{Buckley:2014ana}, along with the corresponding value for
$\alpha_s$.  The masses of the Higgs boson and the top quark have been
fixed, as in the virtual amplitude, to $m_h=125$\,GeV, $m_t=173$\,GeV,
respectively. Their widths have been set to zero.   
Jets are clustered with the
anti-$k_T$ algorithm~\cite{Cacciari:2008gp} as implemented in the
{\tt fastjet} package~\cite{Cacciari:2005hq, Cacciari:2011ma}, with jet
radius $R=0.4$ and a minimum transverse momentum 
$p_{T,min}^{\rm{jet}}=20$\,GeV.  The scale uncertainties are
estimated by varying the factorisation/renormalisation scales
$\mu_{F}, \mu_{R}$. The scale variation bands 
represent scale variations around the central scale $\mu_0 =\mhh/2$, with
$\mu_{R,F}=c_{R,F}\,\mu_0$, where $c_R=c_F= \{0.5,0.5\}$ and $c_R=c_F= \{2,2\}$.
For the case $\lambda=\lambda_{\mathrm{SM}}$ we checked that the bands
obtained from these variations coincide with the bands resulting from
7-point scale variations. The PDF uncertainties have been studied in
\cite{deFlorian:2016spz} and found to be in general considerably smaller than the scale uncertainties.

\subsection{Total cross sections at different values of the trilinear coupling}

In Table \ref{tab:sigmatot} we list total cross sections at 13, 14 and 27\,TeV for various values of the trilinear Higgs coupling $\lambda$. 
\begin{table}[htb]
\begin{center}
%\setlength{\extrarowheight}{3.0pt}
\begin{tabular}{| c | c | c |c|c|}
%\Xhline{2\arrayrulewidth}
\hline
&&&&\\
$\lambda_{\mathrm{BSM}}/\lambda_{\mathrm{SM}}$ & $\sigma_{\rm{NLO}}@13 \mathrm{TeV}$\,[fb]& $\sigma_{\rm{NLO}}@14 \mathrm{TeV}$\,[fb] & $\sigma_{\rm{NLO}}@27 \mathrm{TeV}$\,[fb] &K-factor@14TeV\\
&&&&\\
\hline
-1& 116.71$^{+16.4\%}_{-14.3\%}$  & 136.91$^{+16.4\%}_{-13.9\%}$& 504.9 & 1.86 \\
\hline
0& 62.51$^{+15.8\%}_{-13.7\%}$ & 73.64$^{+15.4\%}_{-13.4\%}$& 275.29& 1.79  \\
\hline 
1& 27.84$^{+11.6\%}_{-12.9\%}$ & 32.88$^{+13.5\%}_{-12.5\%}$&127.7$^{+11.5\%}_{-10.4\%}$ &1.66\\
\hline
2 & 12.42$^{+13.1\%}_{-12.0\%}$ & 14.75$^{+12.0\%}_{-11.8\%}$ &  59.10 & 1.56 \\
\hline
2.4& 11.65$^{+13.9\%}_{-12.7\%}$ & 13.79$^{+13.5\%}_{-12.5\%}$& 53.67 & 1.65 \\
\hline
3& 16.28$^{+16.2\%}_{-15.3\%}$ & 19.07$^{+17.1\%}_{-14.1\%}$ & 69.84 & 1.90 \\
\hline 
5& 81.74$^{+20.0\%}_{-15.6\%}$  & 95.22$^{+19.7\%}_{-11.5\%}$& 330.61 & 2.14 \\
\hline 
\end{tabular}
\end{center}
\caption{Total cross sections for Higgs boson pair production at full NLO. The given uncertainties are scale uncertainties. 
\label{tab:sigmatot}}
\end{table}
Table~\ref{tab:sigmatot} also shows that the K-factors vary substantially as functions of the trilinear coupling.
This fact is illustrated in Fig.~\ref{fig:Kfacvariation}, showing that the K-factor takes values between 1.56 and 2.16
if the trilinear coupling is varied between $-5\leq \chhh\leq 12$.

\begin{figure}[htb]
  \centering
    \includegraphics[width=0.4\textwidth]{plots/Kfac_varlambda.pdf}
%  \includegraphics[width=\textwidth]{plots/}
%    \caption{\label{fig:lambda_large_27}}
\caption{Variation of the NLO K-factor with the trilinear coupling at $\sqrt{s}=14$\,TeV.}
\label{fig:Kfacvariation}
\end{figure}


\subsection{Differential cross sections}

In Figs.~\ref{fig:lambda_small} and ~\ref{fig:lambda_large} we show the $\mhh$ distribution for various values of $\chhh=\lambda_{\mathrm{BSM}}/\lambda_{\mathrm{SM}}$. 
The ratio plots show the ratio to the result with $\lambda_{\mathrm{SM}}$. 
A characteristic dip develops in the $\mhh$ distribution around $\chhh=2.4$, which is the value of maximal destructive interference between diagrams containing the trilinear coupling (triangle-type constributions) and ``background" diagrams (box-type contributions).
Therefore we provide results for a denser spacing of $\chhh$ values around this point.

\begin{figure}[htb]
 \begin{subfigure}[t]{0.495\textwidth}
\includegraphics[width=\textwidth]{plots/{NLO_cHHH_1_2_2.4_0_mHH-paper}.pdf}
%    \vspace{\TwoFigBottom em}
\caption{\label{fig:lambda_small}}
\end{subfigure}
\hfill
\begin{subfigure}[t]{0.495\textwidth}
    \includegraphics[width=\textwidth]{plots/NLO_cHHH_1_-1_3_5_mHH-paper.pdf}
\caption{\label{fig:lambda_large}}
\end{subfigure}
\caption{Higgs boson pair invariant mass distributions for various
  values of $\chhh$  at $\sqrt{s}=14$\,TeV. The uncertainty bands are from
  scale variations as described in the text.}
\label{fig:lambdavar14TeV}
\end{figure}

\begin{figure}[htb]
 \begin{subfigure}[t]{0.495\textwidth}
\includegraphics[width=\textwidth]{plots/{NLO_cHHH_1_2_2.4_0_ptH}.pdf}
\caption{\label{fig:lambda_small_pTH}}
\end{subfigure}
\hfill
\begin{subfigure}[t]{0.495\textwidth}
    \includegraphics[width=\textwidth]{plots/NLO_cHHH_1_-1_3_5_ptH.pdf}
\caption{\label{fig:lambda_large_pTH}}
\end{subfigure}
\caption{Higgs boson transverse momentum distributions for various values of $\lambda$ (relative to $\lambda_{\mathrm{SM}}$)  at 14\,TeV.
\label{fig:lambdavar14TeV_pTH}}
\end{figure}

\subsection{Discussion of parton shower related uncertainties} 
 
In this section we show distributions of the transverse momentum of the Higgs boson pair, where NLO is the first non-trivial order.
We compare the \pythia~\cite{Sjostrand:2014zea} and \herwig~\cite{Bellm:2017bvx} parton showers, applied directly to the \powheg{} Les Houches events (LHE).
In the \herw case, we also compare the default shower (the angular-ordered $\tilde{q}-$shower) with the dipole shower.
In addition, we assess the uncertainties stemming from the matching and show results where the \herw parameter {\tt HardScale} is varied between $0.5$ and $2$. For all shower algorithms considered, the default tune of the corresponding version is used. Multiple-parton interactions (MPI) and hadronisation are switched off. The \hdamp{} parameter in \powheg{} is set to $\hdamp=250$\,GeV.

\begin{figure}[htb]
 \begin{subfigure}[t]{0.495\textwidth}
\includegraphics[width=\textwidth]{plots/{NLO_PP8_PH7_PH7D_cHHH_1_ptHH}.pdf}
 \caption{\label{fig:lambda_ptHH_cHHH1}}
\end{subfigure}
\hfill
\begin{subfigure}[t]{0.495\textwidth}
    \includegraphics[width=\textwidth]{plots/{NLO_PP8_PH7_PH7D_cHHH_2.4_ptHH}.pdf}
\caption{\label{fig:lambda_ptHH_cHHH2.4}}
\end{subfigure}
\caption{Transverse momentum of the Higgs boson pair for the fixed-order NLO calculation and all three shower setups at 14\,TeV for $\lambda=1$.
\label{fig:lambdavar14TeV_pTHH_showers}}
\end{figure}

In Figs.~\ref{fig:lambda_ptHH_cHHH1} and \ref{fig:lambda_ptHH_cHHH2.4}, the transverse momentum $\pthh$ of the Higgs boson pair system is shown for the fixed-order NLO case, as well as for the \pythia (PP8) and both \herw showers (angular-ordered PH7-$\tilde{q}$, and PH7-dipole) for  $\chhh=1$ and $\chhh=2.4$. In all cases, the \pyth and \herw showers agree very well in the small-$\pthh$ range, but then start to deviate at $\pthh \approx 100$\,GeV. Where both \herw showers converge back to the fixed-order calculation at high-$\pthh$, the \pyth shower produces much harder additional radiation and the ratio to the fixed-order result stagnates at $\sim 2.0$ over the full range. On the other hand, the \herw $\tilde{q}-$ and dipole showers give very similar results.
